\chapter{Bevezetés}
Dolgozatomban be fogom mutatni, hogy miként ismerkedtem  meg a természeti eszközökkel, ezek  miként jelentkeznek a textil területen tradicionálisan és napjainkban.
Technológiai megoldásokat mutatok be, melyek mindegyikében a természetes színezékek fontos szerepek játszanak.
Dolgozatomban kitérek két földrajzi régióra ezen, belül is elsősorban Európa és azon belül is legfőképpen Magyarország, hogy  miként jelentek meg a kelmesfestéshez szükséges alapanyagok és technikák és hogy hogyan alakult ez a ma ismert kékfestésé. Ezzel párhuzamban bemutatom a Japánban kialakult Shibori technikát, hogy ott milyen anyag és festési technikát alkalmaznak, és milyen mintákat használnak, milyen módon hozzák létre ezeket a mintákat.
Milyen különbségek merülnek fel, a Magyarországi kékfestészet és a Japán Shibori között, mi milyen módon készítik elő az anyagot a festéshez és a mintázáshoz.
A hozzá tartozó eszközöket miként és hogy használjuk, 
ez a két technológia miként dolgozza fel, a természet általa adott eszközöket és inspirációt.
Miként jelenek meg a textiliparban legfőképpen az indigóval való festés, kékfestés esetében.
Bemutatok olyan magyar kortás művészeket akik, a saját kézműves technikájuk által saját és újra gondolt adaptációkat hoztak létre 