\chapter{Bevezetés}
Dolgozatomban be fogom mutatni, hogy miként ismerkedtem  meg a természeti eszközökkel, ezek  miként jelentkeznek a textil területen tradicionálisan és napjainkban.
Technológiai megoldásokat mutatok be, melyek mindegyikében a természetes színezékek fontos szerepek játszanak
Dolgozatomban kitérek két földrajzi régióra ezen, belül is elsőként Japánra ahol a Shibori technikát mutatom be, hogy ők milyen anyag és festési technikát alkalmaznak, és milyen mintákat használnak, milyen módon hozzák létre ezeket a mintákat .
Milyen különbségek merülnek fel, a Magyarországi kékfestészet és úgye a Japán Shibori között, mi milyen módon készítjük elő az anyagot a festéshez és a mintázáshoz .
A hozzá tartozó eszközöket miként és hogy használjuk , 
Ez a két technológia miként dolgozza fel, a természetet általa adott eszközöket és az általa adott inspirációt.
Miként jelenek meg a textiliparban legfőképpen az indigóval való festés, kékfestés esetében .
Bemutatok olyan magyar kortás művészeket akik, a saját kézműves technikájuk által saját és újra gondolt adaptációkat hoztak létre 
