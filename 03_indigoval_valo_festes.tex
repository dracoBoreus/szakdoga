
\chapter{Az indigóval való festés}
Az indigó név Római indicum-ból származik, ami Indiai terméket jelent.
Ez a megnevezés bizonyos értelemben téves név mert az indigót tartalmazó növényeket a történelem során a világ több pontján is termesztették, mint például Kína, Jáva, Japán és Közép-Amerika.  

Az 13. százelején amikor Marco Polo haza érkezet ázsiai utazásáról, amelynek során rájött, hogy az Europában hőn áhított és nagyon kereslet indigó nem egy ásvány, hanem egy növényekből származó festékanyag és magával hozott egy palántát új színtörtbe az Európai divatba. Ezt követően már Európában is beszerezhetővé vált.

Majd a 16. században Portugália és Spanyolország bonyolította le az indigó kereskedelem nagy részét, mondhatni Indigó nagyhatalmakká lettek.
Ezt követően a hollandok sajátították ki maguknak mert elkezdték nagyobb mennyiségben szállítani az indigót köszönhetően a modernebb és fejlettebb hajó flottáiknak. Ennek következményeként tönkre ment az Európai csülleng termelés. A legjelentősebb mennyiség gyártást Indiában kezdték el 1600 évek végén sok indigót hoztak be Európa.
 
\section{Technologiája}
 \textbf{Növényi indigó}
 Europában az indigó előállításánál festőfűként nevezett (festő csüllenget) használtak elsődlegesen. Ez a egy kétnyári fű fajta ami az év első felében levél szerű rozettákat nőveszt.
 Amikor az indigót kezdik el termeszteni, akkor learatják ezeket a rozetta szerűségeket. A következő évben virágkocsányokat és magokat hoz létre.
 A középkor tájékán fontos gazdasági szerepe volt a festő fűnek Europában, ez hozta a legtöbb jövedelmet.  
 A hagyományos indigó festék előállításánál a festő fű leveleit péppé zúzták labda formákat készítettek belőlük ezeket több hétig szárították és ezt majd egy fermentációs, erjesztési folyamat követte majd a feldolgozás további lépéseiben is többször újra és újra összezúzták. \cite{tiborindigokemia}
 Sajnos a festő fű nagyon szennyezett volt, és ezért csak egy nagyon halvány kék színt tudót adni.
 Ezzel ellentétben a trópusokról származó indigó sokkal jobb minőségű volt és mélyebb gazdagabb kék színt hozott.
 A növényi anyag kapott egy vizes áztatást, utána elkezdték az erjesztési folyamatot és ezt követte a levegőn történő oxidációs folyamat. Maga a növény nem tartalmazta az indigót, hanem azokat az alap vegyületeket melyek a fermentáció során átalakulva hozzák létre a kívánt pigmentet.

\section{Földrajzi varianciája}

\subsection{Magyarországon}

Magyarországon a 18. század közepétől volt elterjedve ez a mesterség.

A Nyugat-Magyarországon lévő manufaktúrákban egyre több külföldi munkás fordult meg, ezen részén az országnak nagy fejlődés vette kezdetét .
A 18. század tájékán Mária Terézia idején kezdődött el Magyarországon az indigó és csülleng fű termesztés.

Azért próbálkozott Magyarország a termesztéssel mert szerették volna csökkenteni az Európát elárasztó indiai és türingiai festéket.
Ebben az időben komoly gazdasági alapja volt Magyarországon, több mint 300 falu foglalkozott a csülleng fű termesztésével  és feldolgozásával.
Néhány vidéki városban a 19. század idején az ottani igényeket figyelembe véve készítették el a termékeket legyen akár szó lakástextilről vagy egyéb felhasználásról.

\subsection{Európában}
A kékfestett textil kereskedelem fő központja 18. században Franciaország  és Németország volt.

\vspace*{3 mm}
Franciaországban a kékfestést technológiája eltért a többi Európai országhoz képest. Az ottani módszer jobban hasonlított a batikoláshoz. A textilanyag felületét viasz réteggel vonják be, arra a részre, ahol nem szeretnék, hogy befogja a festék és ezzel a módszerrel sokkal tisztább és élesebb dekoratív mintákat tudtak elérni ennek a viaszos alkalmazásnak a segítségével.
Mintakincsük sokkal aprólékosabb részlet gazdagabb inkább levélszerű mintákat alkalmaznak, ezek közt elvétve fel tűnik egy két virág motívum mely apró és elhanyagolható a teljes mintában.
A minta fa készítéshez gyümölcsfa fajtákat használtak fel mert sokkal lágyabbnak és puhábbnak tartják a véséshez és kaparáshoz. Az általuk előállított anyagok több színűek és esetenként a minták is színesek.
Például a vörösszín eléréshez festő fűt alkalmaztak a festő buzér - Garance - Rubia tinctorum  ez a nővény alapozta meg a vörös szín használatát, 
a sárga és a zöld szín előállítása a festő csülleng feldolgozásának részfolyamatainál mint melléktermék jött elő.

\vspace*{3 mm}
Németországban három technológia volt elterjedve a Tartalék (Reservedruck), közvetlen (Direktdruck), maratás (Ätzdruck). Az alap textilt először fehérítették, szárították és végül hengerelve (Mangeln) vasalták és keményítették.

A tartalék nyomás során fehér mintát hoznak létre kék alapon. Az eljárás során a gumiból hozzák létre a nyomatot. Ezt követően a nyomatot (Reservage) bevonják egy festék taszító anyaggal és a minta nyomás száradást követően történhet. A festék összetétele: fehér dohány, réz-szulfát, réz-acetát és egéb titkos összetevők. Pasztell vagy német indigónak hívták ezt a festéket. A festés után a mintát hígítót kénsavval távolítják el.

Ezért ez egy festési eljárás, nem egy nyomtatási folyamat. A "tartalék" kifejezés arra a tényre utal, hogy a kiválasztott minta a festés során kimarad.

A közvetlen nyomtatásban a színt az előkészített fehér szövetre nyomtatják a nyomdarúddal. A szín közvetlenül az anyag felületére kerül, és barna színként jelenik meg. A száradás után a szövet egy fürdőbe helyezik, amelyben a barna tinta kémiai reakcióval élénkkékre változik. A szövet végül felfőzik, újra hengerlik, ezt követően használatra kész. A nyomtatást nagyon óvatosan kell elvégezni, mivel a hibákat nem lehet kijavítani.

A maratás folyamata egy kék színű alap szövet is nyomtatható maró anyaggal (maró folt), ami kék alapon fehér mintázatot hoz  létre. Ez a technika alapvetően Hollandiából származik, de máshol is elterjedt az eljárás.

\subsection{Ázsiában}
A Shibori egy a 8. századból származó japán textil festési technika \cite{shibori} amelynek lényege, hogy különböző kötözési, csomózási eljárásokkal állítanak elő tetszőleges mintázatot.

A Shibori számtalan más japán alap technikát foglal magában, iylen technikák például a miura, a kumo, a nui, mindegyikben közös, hogy valamilyen kötözési módszer vagy az anyag hajtogatást alkalmazz. 

A kötözés és hajtogatás mellet a szövetett olykor festéktaszító anyaggal is bevonják és ezzel is mintákat hoznak létre.

A teljes minta elő állításának folyamata az alap szövet tisztításával és esetleg fehérítésével kezdődik, ezt követően történik a vasalás és hajtogatás vagy a szövet összefogása és kötözése és felületének kezelése.
Az Shiboriban az, hogy milyen technikát használunk nem csak a saját preferenciánktól függ vagy a vágyott mintától, hanem a befestendő anyag tulajdonságaitól is. Ezen kívül  a különböző festési módszerek kombinálhatunk kidogozottabb, bonyolultabb, részletesebb minták elérése érdekében. 

\subsection{Különböző régiók összehasonlítása}
   Az országok közötti hasonlóság elsősorban az, hogy ugyan azt a festő nővényt használják fel a munkájuk során.
   De technikailg a minták előállítást is más kép működik,és ezért nehéz a minták kőzött hasonlóságot keresni , de a  minták  elkészítéséhez valamilyen téren a természet által inspirálódnak , míg az egyiknél teljesformájában jelenek meg a természeti minta elemek másik részéről még vonallak vagy pontok , körök formájában jelenek meg a minták  
    igazából a minták elkészítés mind a két téren finom érzéket és türelmet igényel.
