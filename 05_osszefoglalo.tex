\chapter{Összegzés} 
A szakdolgozatom irássa közben megismertem, hogy a természet mint eszköz és forrás hogyan valósul meg és ezek milyen hatással voltak/vannak az emberiségre milyen technikák, minták alakultak ki.
Ezeknek az ismerete új inspirációt adott és irányt adtak a saját mintakicsem bővitésére.

Megismerkedtem azokkal az alap módszerekkel amivel az ember felhasználta a természetes szineket és azt, hogy hogyan alakították ki, használták fel és hogy ez mely területeken fordult mind földrajzilag és művészetileg.
Ezek a technikák a legfőképpen a textil művészetekben valósulnak meg az alapanyagok megfestésével, színezésével.

A dolgozat alatt két fő régiót vizsgáltam meg Ázsiát és Európát ezeken belül is elsődlegesen Japánt és Magyarországot. Japánban a shibori technika míg magyarországon  a kékfestő volt.

E két technika bemutatása során meg ismerkedtem a folyamatokkal melyekkel a textíliákat színezik és ahogy ezeken kialakítják a mintázattokat.
Bemutattam olyan kortás művészeket akiki  két hagyományos kézműves technika álltal inspirálódtak és saját adaptációt valósítottak meg.