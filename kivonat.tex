\thispagestyle{plain}
\begin{center}
    \Large
    \textbf{SZAKDOLGOZATI KIVONAT}
\end{center}
\textbf{1. A szakdolgozat alapvető kulcsszavai}
\begin{enumerate}[label=\alph*)]
	\item Minta
	\item Természet
	\item Textílművészet
\end{enumerate}
\textbf{2. A szakdolgozatban használt legfontosabb források}
\begin{enumerate}[label=\alph*)]
	\item Domonkos, O. (1981).A magyarországi kékfestés
	\item Antal, J. (1963).A móra ferenc múzeum évkönyve(B. Alajos, Ed.). Móra Ferenc Múzeum.
	\item Meller, S., \& Elffers, J. (2002). Textile designs: 200 years of patterns for printed fabrics arranged by motif, colour, period and design. Thames \& Hudsonk
\end{enumerate}
\newpage
\textbf{3. A szakdolgozat magyarnyelvü összefoglalása}
\vspace{0.2 cm}
\\
Dolgozat témám a természet megjelenítése a mintatervezésben.
Az egyik kutatási területem, az hogy a természeti mintaelemek miként jelenek meg a művészetben azon belül is a Textil művészetben, milyen alapanyagokat, eszközöket  hassználtak fel az előállításaik során a természet által és ez hogyan jelenik meg a kékfestészetben.
Bemutatok néhány magyar textilművészt akik ezen a területen és a természt által inspirálodav modern adaptációkat valósítotak meg.

\vspace{2 cm}

\textbf{4. A szakdolgozat angolnyelvü összefoglalása}
\vspace{0.2 cm}
\\
\begin{otherlanguage}{english}
My dissertations topic is the display of nature in pattern design.
One of my research areas is how natural patterns  appear in art, including the raw materials and tools used frm nature during their productions, and how this appears in traditional blue painting.
I also present Hungarian textile artists who have made modern adaptations in this field and are inspired by nature. 
\end{otherlanguage}
\newpage