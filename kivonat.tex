\thispagestyle{plain}
\begin{center}
    \Large
    \textbf{SZAKDOLGOZATI KIVONAT}
\end{center}
\textbf{1. A szakdolgozat alapvető kulcsszavai}
\begin{enumerate}[label=\alph*)]
	\item Minta
	\item Természet
	\item Textílművészet
\end{enumerate}
\textbf{2. A szakdolgozatban használt legfontosabb források}
\begin{enumerate}[label=\alph*)]
	\item Domonkos, O. (1981).A magyarországi kékfestés
	\item Antal, J. (1963).A móra ferenc múzeum évkönyve(B. Alajos, Ed.). Móra Ferenc Múzeum.
	\item Meller, S., \& Elffers, J. (2002). Textile designs: 200 years of patterns for printed fabrics arranged by motif, colour, period and design. Thames \& Hudsonk
\end{enumerate}
\newpage
\textbf{3. A szakdolgozat magyarnyelvü összefoglalása}
\vspace{0.2 cm}
Dolgozat témám a természet megjelenítése a mintatervezésben .
Az egyik kutatási területem a természeti mintaelemek miként jelenek meg a művészetben azon belül a Textil művészetben milyen alapanyagokat , eszközöket  hassználtak fel természet által .
ez hogyan jelenik  meg a kékfestészetben is .
Bemutatok néhány olyan magyar textilművészt akik ezen területen és a természt által inspirálodav csodás munkákat hozztak létre

\vspace{2 cm}

\textbf{4. A szakdolgozat angolnyelvü összefoglalása}
\vspace{0.2 cm}
\begin{otherlanguage}{english}
This is the abstract what is abstract
% ide írd az angol abstractot, különben magyar elválasztást használ
\end{otherlanguage}
\newpage