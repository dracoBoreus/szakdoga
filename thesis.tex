\documentclass[fontsize=12pt, appendixprefix=true]{scrreprt}
% appendixprefix: hogy odaírja, hogy "Függelék A", ne csak "A"
\usepackage[english, magyar]{babel}                        % nyelvi csomag
\usepackage[T1]{fontenc}                                   % ékezetes bet?knél is legyen automatikus elválasztás
\usepackage[utf8]{inputenc}                                % ékezetes bet?k kezelése
\usepackage{mathptmx,times}                                      % alapértelmezett bet?típus ne legyen pixeles
\usepackage{mathtools}                                     % képletekhez kell
\usepackage[style=ieee, backend=biber]{biblatex}           % bibliográfia
\addbibresource{thesis.bib}
\usepackage{graphicx}                                      % képek beszúrása
\usepackage[export]{adjustbox}                             % ez a logó pozicionálásához kell
\usepackage[margin=2.5cm, bindingoffset=0.5cm]{geometry}  % margók
\usepackage[onehalfspacing]{setspace}                      % másfeles sorköz
\usepackage[hidelinks, unicode, pdfusetitle]{hyperref}     % kattintható tartalomjegyzék és hivatkozások
\usepackage{bookmark}                                      % PDF könyvjelz?k
\usepackage{csquotes}                                      % a bibliográfiában megfelel?en legyenek formázva az idéz?jelek
\setcounter{tocdepth}{2}
\DeclareQuoteAlias{dutch}{magyar}

% Kódrészletekhez ajánlom
\usepackage{listings, scrhack}
\usepackage{sourcecodepro} % egy jó bet?típus
\lstset{captionpos=b, numberbychapter=false, basicstyle=\ttfamily, showstringspaces=false, columns=fullflexible}
\renewcommand\lstlistingname{kódrészlet}
\makeatletter
\renewcommand\fnum@lstlisting{\ifx\lst@@caption\@empty\else\thelstlisting.~\fi\lstlistingname}%
\makeatother

\titlehead{\includegraphics[width=1.0\linewidth,height=0.5\textheight]{logo}}
\subject{A természet megjelenítése a minta tervezésben}
\author{Sárkány Lili\\Kézműves Tárgykúltúra}
\title{Szakdolgozat}
\date{2021}
\publishers{Témavezető:\\Papp Anett}

% Nyilatkozathoz két parancs definíciója
\newcommand{\pushtobottom}{\vspace*{\fill}}
\newcommand{\signatureline}[1]{\begin{flushright}
	\vspace*{.5cm}\par\noindent\makebox[2.5in]{\hrulefill}
	\par\noindent\makebox[2.5in][c]{#1}
	\end{flushright}
}

\begin{document}
\maketitle

Alulírott ...
\pushtobottom
\signatureline{Aláírás}

\tableofcontents
%\newpage
%\section*{Kivonat}
%
%\newpage
%\section*{Abstract}
%\begin{otherlanguage}{english}
%% ide írd az angol abstractot, különben magyar elválasztást használ
%\end{otherlanguage}

\chapter{A természet mint eszköz és forrása}
\section{Alap fogalmak}
\section{A természet mint művészeti inspiráció}
\section{A természet mint eszköz a művészetben}
\subsection{Színanyagok}
\subsection{Eszközök}

\chapter{Az indigóval való festés}
\section{Technologiája}
\section{Földrajzi varianciája}
\subsection{Magyarországon}
\subsection{Ázsiában}
\subsection{Különböző régiók összehasonlítása}

\chapter{A kékfestés}
\section{Előzménye és Történelmi áttekintése}
\section{Technológiája}
\subsection{Alapanyaga}
\subsection{Folyamata}
\section{A természet mint inspiráció}
\section{Minta dúcok}
\subsection{Alap minták}
\subsection{Minta fejlődés}
\subsection{Egyébb fő minták motívumok}
\section{Hagyományos kékfestő műhelyek}
\section{Egyéb műhelyek}
\subsection{Kortás megoldások}
\subsection{Adaptációk}

\chapter{Összefoglalás} 


\printbibliography
\renewcommand{\appendixname}{Függelék}  % különben "Appendix"-et ír
\appendix
\chapter{Mellékletek listája}

\end{document}