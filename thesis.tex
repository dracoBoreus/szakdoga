\documentclass[fontsize=12pt, appendixprefix=true]{scrreprt}
% appendixprefix: hogy odaírja, hogy "Függelék A", ne csak "A"
\usepackage[english, magyar]{babel}                        % nyelvi csomag
\usepackage[T1]{fontenc}                                   % ékezetes bet?knél is legyen automatikus elválasztás
\usepackage[utf8]{inputenc}                                % ékezetes bet?k kezelése
\usepackage{mathptmx,times}                                      % alapértelmezett bet?típus ne legyen pixeles
\usepackage{mathtools}                                     % képletekhez kell
\usepackage[style=ieee, backend=biber]{biblatex}           % bibliográfia
\addbibresource{thesis.bib}
\usepackage{graphicx}                                      % képek beszúrása
\usepackage[export]{adjustbox}                             % ez a logó pozicionálásához kell
\usepackage[margin=2.5cm, bindingoffset=0.5cm]{geometry}  % margók
\usepackage[onehalfspacing]{setspace}                      % másfeles sorköz
\usepackage[hidelinks, unicode, pdfusetitle]{hyperref}     % kattintható tartalomjegyzék és hivatkozások
\usepackage{bookmark}                                      % PDF könyvjelz?k
\usepackage{csquotes}                                      % a bibliográfiában megfelel?en legyenek formázva az idéz?jelek
\setcounter{tocdepth}{2}
\DeclareQuoteAlias{dutch}{magyar}

% Kódrészletekhez ajánlom
\usepackage{listings, scrhack}
\usepackage{sourcecodepro} % egy jó bet?típus
\lstset{captionpos=b, numberbychapter=false, basicstyle=\ttfamily, showstringspaces=false, columns=fullflexible}
\renewcommand\lstlistingname{kódrészlet}
\makeatletter
\renewcommand\fnum@lstlisting{\ifx\lst@@caption\@empty\else\thelstlisting.~\fi\lstlistingname}%
\makeatother

\titlehead{\includegraphics[width=1.0\linewidth,height=0.5\textheight]{logo}}
\subject{A természet megjelenítése a minta tervezésben}
\author{Sárkány Lili\\Kézműves Tárgykúltúra}
\title{Szakdolgozat}
\date{2021}
\publishers{Témavezető:\\Papp Anett}

% Nyilatkozathoz két parancs definíciója
\newcommand{\pushtobottom}{\vspace*{\fill}}
\newcommand{\signatureline}[1]{\begin{flushright}
	\vspace*{.5cm}\par\noindent\makebox[2.5in]{\hrulefill}
	\par\noindent\makebox[2.5in][c]{#1}
	\end{flushright}
}

\begin{document}
\maketitle

Alulírott ...
\pushtobottom
\signatureline{Aláírás}

\tableofcontents
%\newpage
%\section*{Kivonat}
%
%\newpage
%\section*{Abstract}
%\begin{otherlanguage}{english}
%% ide írd az angol abstractot, különben magyar elválasztást használ
%\end{otherlanguage}

\chapter{A természet mint eszköz és forrása}
A természeti motívumok és eszközök már az ősidők óta részét képezi az emberi művészetnek. \cite{domonkos1981magyarorszagi} A korai barlang rajzok motívumaiban és elkészítésükhöz felhasznált \cite{tiborindigokemia} anyagoknál az elsődleges forrás a természet volt.
\section{Alap fogalmak}
\section{A természet mint művészeti inspiráció}
\section{A természet mint eszköz a művészetben}
\subsection{Színanyagok}
\subsection{Eszközök}

\chapter{Az indigóval való festés}
Nem egyszerű egy olyan részt bemutatni a témával kapcsoltban ami több 1000 éves múltre tekint vissza .
Az indigó szinte már körbe járta a világot 

\section{Technologiája}
Még mielőtt nem volt az indigó akkor évszázadokon át egy természetes anyagot használtak az nem más volt mint festőcsülleng
Ez a növényi indigó  amit az első évben levélszerűrozetáket termel , aztán az év másik részében mar ilyen virágkocsányokat , magokat hozz létre.
Európában  a középkor idején nagyon fontos volt mert a termelöknek ez hozzta a legtöbb jövedelmet 
Hagyományos indigfesték	eLöálitásánál a leveleket össze zuzták aztán össze gyurták ilyen gólyó formának hetekig nedvesiteték és erjeszteték .
következö folyamat az volt hogy az össze zuzás után meg tisztitoták az anyagot mieloött elkezdték használni szorosan ki préselték mert nagyon szenyezet volt és joval világosabb kék szint adott mint az indigó volt de hát sajnos amig nem jöttbe az országba az indigó ezzt kellet használniuk


\section{Földrajzi varianciája}
Az indigót a világ több pontján is termeszteték  Indián kivül mint például Jáva .Japán Közép-aAmerikát és Dél- Amerikát beleértve 
1200-as években mikor Marco Polo haza érkezet ázsiai utazásáról , rájött hogy az indigó nem egy ásvány hanem egy növényekböl származó festékanyag.
Késöbbiekben már beszerezhető volt kisebb mennyiségeben Európában .
De a 16 század bann a portugália és spanyolország bonyolitota le az indigokereskedelmet.
Aztán a Hollandok sajátitoták ki maguknak a mert elkzték szállitani az indigot és ezzel tönkre téve a csüllengtermelö gazdaságot 
jeletösebb mennyiség gyártást indiában kezdték el 1600 évek végén sok indigót hozztak be Europába 



\subsection{Magyarországon}
A kékfestö mesterség fénykorábban 400 kékfestö votl Magyarországon ebböl 300 Dunántulon volt ebböl több fen marado 100 bol fen marado 60-70 kékfestö duna tiszta közén telepedt le és talan 5-10 jutott tiszán tulra a Tothok lakta vidékeken telepedtek  le Békéscsabán , Szarvasobn , Debrecenben 
10 több kék festö nem voltak város szerte 
nagyon sokan voltak akiki egyedül végztek el ezt a munkát mert nem volt elég helyuk hogy többen is elvégzék a feladatokat .
Mint pélául Pápán az egy sokkal nagyobb mühely volt és ott körübelül 30 dolgoztak 

A 18 század tájékán Mária Terézia iddején kezdötött el magyarországona in digó és csüllengtermesztés.
Azért probálkozott Magyarország a termesztésel mert szereték volna csökkentei az indiai és türingiai festéket
több mint 300 falu foglalkozott a csülleng termesztésével  és feldolgozásával 
Magyarország 

\subsection{Ázsiában}
\subsection{Különböző régiók összehasonlítása}

\chapter{A kékfestés}
\section{Előzménye és Történelmi áttekintése}
Kékfestészet elődjének kelmefestés és a textilnyomást tartjuk mely technológiák 16 században már  léteztek .
Gyakorlatban használtak hozzá kék szín elérése érdekében festőcsüllenget (Isatis tinctoria) és az indigót(Indigofera tinctoria használták nyomódúcok használatával mintás kelméket hozztak létre 

Magyarországon már a 16 században is léteztek Festömühelyek de igazábol 17 század végére jutott el odáig hogy meghonosodott ez az egész iparág itt magyarországon 

18 század közepe tájan már ország minden  városában vagy akkár mezővárosában elöfordultak festőmesterek akik az ottani igényket felmérve készitettek  anyagokat probáltak a mesterek a sajátoságaikra törekedni hogy mindegyik anyag minta külömbözön a városoktola 
A mesterekre nagy feladatok vártak mert a megszokot festési eljárásokról a csüllengfestéröl áttkellet térni a az indigóval való festésre mert az indigó sokkal szebb szinüvolt és pigmentáltabb volt és sokkal tartosabb volt.
Az ottani emberek nagy igényt tartottak jo minöségü anyagokra 


\section{Technológiája}
A kékfestési technológia egy nagyon fontos szemponton alapul amit nem más hogy neveznek  ún gátlónyomásnak nevezik
Ez igazából egy gátló szer ami egy adott szint elszigetel a szövettöl .
persze ahova nem került a szerböl azt a részekett befogja. 

\subsection{Alapanyaga}


\subsection{Folyamata}
kékfestészet  munkafolyamata ugy zajlott le egy műhelyben . hogy a nyersanyagot magánszemélytöl vásárolták vagy maga rendelö vitte magával az anyagot és még a mintát is ki tudta választanoia  a mintakonyvböl
Budapesten voltak olyan emberek hogy csak azzal foglalkoztak a kékfestésel foglakozo mesterekt látták el anyagókal
Az első  folyyamt amit kezdenek az a anyagokal .hogy ki mossák öket szódás vízben ez körübelül két órát vesz igénybe .
 A szódás kifözést azért végzik el hogy az anyagböl ki oldoldjon egy gyantás anyag és ez a folyamat után  az anyag kifehéredik
 Miután ez a folyamt befejezödött akkor következik a hideg vizes öblités majd léces száritora helyezik majd néhány óra alatt meg szárad ez a folyamt nyáron gyorsanbb télen egy kicsit lassab folyamat 
Mikor már megszáradt az anyag egy kap egy vékony keményitést ami elkökésziti a mintázáshoz 
Mintázás régies nevén Tarkázás .
Tarkázásnak emlitett kézi minta készités  párnázott asztalon készül.
Amire több rétegben pokrocokat helyeztek majd a végén molinoval befedték igy az asztal felülete 
Igy az asztal felülete puha lett .
Az asztal mellet helyezték el Únt és az ún-ba mártoták bele a minta ducokat.
A viaszosvászon belsö felületére kent fel a mintaázó pépet az Ún 
ládát feltőltőték sürü keményitövel ami a sasit puhán tartotta .
Amit általábban használunk hozzá  fehér nyomo pépet annak alkotorészei ólómnitrát, ólomacetát . gumiarábikum . kékő.timsó, rézgálic .
Ezzeket össze fözzik és miután kihült egy hétre rá használják csak az anyagot 
A festék pépet  maga a mester szokta elkésziteni általában ez a festék pép akkár ávekig is elég szokot lenni a mester az gondolják hogy egy pép minél régrbbi egy nyomo pép annál fehérebbé teszi a mintát
A kékfestö a minta  ducot a nyomopépbe nyomja majd és a ducot vászonra ileszti pontosan nehogy elcsuszon és majd erösen lenyomja hogy minden pontra kerüljön a pépböl és azt a részt nem fogja be az indigófesték .
Mintázás nagy figyelmet és turelmet igényelt 
Altalában a kékfestök  ,120-130 méter anyagot meg mintáztak de az napi 13-14 órás mmunka folyamt volt -
Miután a minta nyomás befejezödött  párnap volt még megszáradt a nyomatés ezzek utén következet a festés .
Az anyagot egy ráfra akasztjak   amÍ  igazábol egy vaskerék érre akasztják fel az anygot és és ugy nézi ez az anyag a ráfon és olyan az esése az anyagnak ezzen a vaskereken mintha egy rakott szoknya lenne .
következö teendö peddig nem  más  lengedik  az indigó festékel teli kádszerüségbe és  félóránként fel engedig és körűbelül - percet hagyák a levgön hogy az indigó oxidálodjon .
Amikor elsönek felhúzúkk az anyagot a festékböl akkor ilyen sárgás zöld szine van aztán megint eltelt egy kis idö akkor ilyen sötétzöld szine lesz az útólö elötti merités után világos kék színe lesz 
miután egyre többet éri a levegö  megkapja azt a szép sötétkék színt 
Miútán az anyag elérte a kivánt színt akkor következik a száritás ezután miután meg száradt akkor kap egy kénsavas fürdöt ennel a folyamtnál az történik hogy kioldodik a mintázásnál beleragadt nyomopép és ezzek után nagyon szépen elöjönna minta és utána egy sima vizben is kiöblitik és hagyák meg száradni , kikeményitik az anyagot 
















\section{A természet mint inspiráció}
\section{Minta dúcok}


\subsection{Alap minták}
\subsection{Minta fejlődés}
\subsection{Egyébb fő minták motívumok}
\section{Hagyományos kékfestő műhelyek}
\section{Egyéb műhelyek}
\subsection{Kortás megoldások}
\subsection{Adaptációk}

\chapter{Összefoglalás} 


\printbibliography
\renewcommand{\appendixname}{Függelék}  % különben "Appendix"-et ír
\appendix
\chapter{Mellékletek listája}

\end{document}