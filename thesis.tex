\documentclass[fontsize=12pt, appendixprefix=true]{scrreprt}
% appendixprefix: hogy odaírja, hogy "Függelék A", ne csak "A"
\usepackage[english, magyar]{babel}                        % nyelvi csomag
\usepackage[T1]{fontenc}                                   % ékezetes bet?knél is legyen automatikus elválasztás
\usepackage[utf8]{inputenc}                                % ékezetes bet?k kezelése
\usepackage{mathptmx,times}                                      % alapértelmezett bet?típus ne legyen pixeles
\usepackage{mathtools}                                     % képletekhez kell
\usepackage[style=ieee, backend=biber]{biblatex}           % bibliográfia
\addbibresource{thesis.bib}
\usepackage{graphicx}                                      % képek beszúrása
\usepackage[export]{adjustbox}                             % ez a logó pozicionálásához kell
\usepackage[margin=2.5cm, bindingoffset=0.5cm]{geometry}  % margók
\usepackage[onehalfspacing]{setspace}                      % másfeles sorköz
\usepackage[hidelinks, unicode, pdfusetitle]{hyperref}     % kattintható tartalomjegyzék és hivatkozások
\usepackage{bookmark}                                      % PDF könyvjelz?k
\usepackage{csquotes}                                      % a bibliográfiában megfelel?en legyenek formázva az idéz?jelek
\setcounter{tocdepth}{2}
\DeclareQuoteAlias{dutch}{magyar}

% Kódrészletekhez ajánlom
\usepackage{listings, scrhack}
\usepackage{sourcecodepro} % egy jó bet?típus
\lstset{captionpos=b, numberbychapter=false, basicstyle=\ttfamily, showstringspaces=false, columns=fullflexible}
\renewcommand\lstlistingname{kódrészlet}
\makeatletter
\renewcommand\fnum@lstlisting{\ifx\lst@@caption\@empty\else\thelstlisting.~\fi\lstlistingname}%
\makeatother

\titlehead{\includegraphics[width=1.0\linewidth,height=0.5\textheight]{logo}}
\subject{A természet megjelenítése a minta tervezésben}
\author{Sárkány Lili\\Kézműves Tárgykúltúra}
\title{Szakdolgozat}
\date{2021}
\publishers{Témavezető:\\Papp Anett}

% Nyilatkozathoz két parancs definíciója
\newcommand{\pushtobottom}{\vspace*{\fill}}
\newcommand{\signatureline}[1]{\begin{flushright}
	\vspace*{.5cm}\par\noindent\makebox[2.5in]{\hrulefill}
	\par\noindent\makebox[2.5in][c]{#1}
	\end{flushright}
}

\begin{document}
\maketitle

Alulírott ...
\pushtobottom
\signatureline{Aláírás}

\tableofcontents
%\newpage
%\section*{Kivonat}
%
%\newpage
%\section*{Abstract}
%\begin{otherlanguage}{english}
%% ide írd az angol abstractot, különben magyar elválasztást használ
%\end{otherlanguage}

\chapter{A természet mint eszköz és forrása}
A természeti motívumok és eszközök már az ősidők óta részét képezi az emberi művészetnek. \cite{domonkos1981magyarorszagi} A korai barlang rajzok motívumaiban és elkészítésükhöz felhasznált \cite{tiborindigokemia} anyagoknál az elsődleges forrás a természet volt.
Akkoriban környezeti adottságok által fedeztek fel a dolgokat mit mire és hogyan tudnák felhasználni ö maguk tapasztalták ki mint például növényeket kövekkel össze préselték és azzal próbáltak lenyomatokat és rajzokat készíteni a barlangok falra vagy maga földet is felhasználtak nyomhagyásra mert 
azt úgy gondolták hogy annál természetesebb anyag  nem létezik mint maga a föld amit vízzel össze kevertek és azt úgy használták fel barlang rajzokon általában kisebb történtetek vagy állat rajzokat láthatunk a barlang rajzokon vagy a sziklákon általában .
állatokat azért jelenítették meg a barlang rajzokon mert ezzel a harciaságukat jelenítették meg  ezek ben a rajzokban.
Természeti anyagokat használva mint ahogy az őskorban is tették 
 


\section{Alap fogalmak}
A munkám során felhasznált  alapvető szükséges fogalmak melyek elengedhetetlenek voltak:
\begin{itemize}
	\item \textbf{Motívum} \\  A művészetekben a legkisebb önálló kifejezőegység, amely a mű során általában ismétlődik.A folklór alkotásainak legkisebb tartalmi egysége, amely átvétel után is felismerhető marad.
	\item \textbf{Organikus} \\ szerves, növényi , állati eredetű
	\item \textbf{Ornamentika} \\ Díszítőelemek összessége; díszítőművészet.
	\item \textbf{Kelmefestés} \\ A kékfestés egy textíliákon alkalmazott színmintázási technológia, amely nevét onnan kapta, hogy a minta eredeti formájában jellegzetesen kék alapon fehér színben jelenik meg.
	\end{itemize}

\section{A természet mint művészeti inspiráció}
A  természet a művészetben számos vizuális formát vesz át .
A művészet utánozhatja a természetet azáltal, hogy vizuálisan lemásolja a tárgyakat, ahogy azok ténylegesen megjelennek a való  világban 
Művészet kinyit olyan ajtókat nekünk a természet felé ami a természet  bonyolultságra  és szépségére figyelmeztet minket,
amit semmi képen nem lehet elszalasztani .
Lehet egy egyszerű kép ami értelmezi a természetet vagy ez lehet egy kihívást jelentő darab kifejezve a  emberi kapcsolat a természettel.
A természet által való művészeti gondolkodás add nekünk egy olyan gondolkodás módot  ami maga a természet által létrejött anyagokkal tárgyakkal való dolgozás  mint például levelek , botok ágak ,víz , kövek és ehhez hasonló természeti tárgyak .
Kreatív módon való felhasználás ami új művészeti tárgyakat készítsünk a környezet , természet  által inspirál ,
természeti anyagokat használva kicsit vissza az őskorba amikor szerves és ásványi anyagokat használtak fel.
Manapság néhány kortás művész messze van attól a folyamtól ténylegesen hogy saját mag hozzon létre festéket vagy bármilyen anyagot .
A művészet a természetből vesz egy hatalmas inspirációt , kezd vissza térni a lényeg a föld maga mint egy anyag alkotás .
egy  kortás művész aki a természet által inspirálódott 
Az első ilyen művész  Andy Goldsworthy  volt aki olyan természetes anyagokat használt fel mint például leveleket és köveket használt fel az alkotásaihoz és ő olyan helypecifukus szobrokat készitett amelyek tükrözik az anyagok és a természet közzöti kapcsolatott
munkájának az elkészítési ideje hosszú folyamat volt.
A kész műveit mindig lefényképezte mert  az időjárás és az idő elemeinek ki voltak téve.
 Ily módon a tényleges műalkotás múlandó.




\section{A természet mint eszköz a művészetben}
\subsection{Színanyagok}
A természetben előforduló pigment anyagok melyek felhasználásra kerülnek a legkülönfélébb művészeti alkotásokban:
Indigó, Bíbor, 
\subsection{Eszközök}
Nyomódúc. textilminták előállítására használt fa mintanyomó eszköz, leginkább puha és keményebb fa fajtákból készült mint például gesztenye fa vagy dió fa vagy tölgy vagy bükk
\chapter{Az indigóval való festés}
Nem egyszerű egy olyan részt bemutatni a témával kapcsoltban ami több 1000 éves múltre tekint vissza .
Az indigó szinte már körbe járta a világot 

\section{Technologiája}
Még mielőtt nem volt az indigó akkor évszázadokon át egy természetes anyagot használtak az nem más volt mint festő csülleng
Ez a növényi indigó , amit az első évben levélszerű  rozettákat  termel , aztán az év másik részében mar ilyen virágkocsányokat , magokat hozz létre.
Európában  a középkor idején nagyon fontos volt mert a termelőknek ez hozza a legtöbb jövedelmet 
Hagyományos indigó festék	előállításánál a leveleket össze zúzták ,  aztán össze gyúrták ilyen golyó formának hetekig nedvesítették és erjesztették .
következő folyamat az volt hogy az össze zúzás után meg tisztították az anyagot mielőtt elkezdték használni szorosan ki préselték mert nagyon szennyezett volt és jóval világosabb kék szint adott mint az indigó volt de hát sajnos amig nem jöttbe az országba az indigó ezt kellet használniuk


\section{Földrajzi varianciája}
Az indigót a világ több pontján is termesztették  Indián kívül mint például Jáva .Japán Közép-aAmerikát és Dél- Amerikát beleértve 
1200-as években mikor Marco Polo haza érkezet ázsiai utazásáról , rájött hogy az indigó nem egy ásvány hanem egy növényekből származó festékanyag.
Későbbiekben már beszerezhető volt kisebb mennyiségeben Európában .
De a 16 század ban a Portugália és Spanyolország bonyolította le az indigó kereskedelmet.
Aztán a Hollandok sajátították ki maguknak a mert elkezték szállitani az indigot és ezzel tönkre téve a csüllengtermelö gazdaságot 
jeletösebb mennyiség gyártást indiában kezdték el 1600 évek végén sok indigót hozztak be Europába 



\subsection{Magyarországon}
A kékfestö mesterség fénykorábban 400 kékfestö votl Magyarországon ebböl 300 Dunántulon volt ebböl több fen marado 100 bol fen marado 60-70 kékfestö duna tiszta közén telepedt le és talan 5-10 jutott tiszán tulra a Tothok lakta vidékeken telepedtek  le Békéscsabán , Szarvasobn , Debrecenben 
10 több kék festö nem voltak város szerte 
nagyon sokan voltak akiki egyedül végztek el ezt a munkát mert nem volt elég helyuk hogy többen is elvégzék a feladatokat .
Mint pélául Pápán az egy sokkal nagyobb mühely volt és ott körübelül 30 dolgoztak 

A 18 század tájékán Mária Terézia iddején kezdötött el magyarországona in digó és csüllengtermesztés.
Azért probálkozott Magyarország a termesztésel mert szereték volna csökkentei az indiai és türingiai festéket
több mint 300 falu foglalkozott a csülleng termesztésével  és feldolgozásával 
Magyarország 
   
.......
középkor idején elsősorban 

\subsection{Európában}

\subsection{Ázsiában}

\subsection{Különböző régiók összehasonlítása}
   Európa nyomódúc <> Batikolás
\chapter{A kékfestés}
\section{Előzménye és Történelmi áttekintése}
Kékfestészet előzményének a kelmefestészetet, és a textilnyomást tartjuk ami egyszínűre színezést és mintázást jelent  mely technológiák már 15 , 16 században is léteztek 
Sok tudást igénylő művészeti ág volt
A középkori kelmefestés kezdetekben a városokban és a kolostorokban működtek különféle anyagok színezésével foglalkoztak mint például gyapjú vagy vászon . 
Akkoriban még Magyarország még nem rendelkezett az Indigóval szóval addig egy másik fajta természetes növény használtak ami  Kék szint adta és úgy neveztek ezt a növényt hogy festő csülleng 
 Ennek a behozatala még a 15 században került sor .

 maga a csülleng fontos szerepet töltőt be a középkortól kezdve 
 Legnagyobb termelő területek Franciaország és Türningában volt .
 A magyarországi mestereknek a beszerzési központ Türnigiában volt 
 A 16 század körül körülbelül 300 falu foglalkozott a csülleng termesztéssel 
 Aki akkoriban a csülleng termesztéssel foglakozott vagyonosabb réteghez tartozott
 de sajnos a 17 század kezdetbe romlani kezdet a mezőgazdasági termesztés ugyan is a holland hajókaravánok behozták kelet indíából az indigót .
 Ezekben az időkben már közel 30 falú foglakozott a termesztéssel és igy hírtelen akik csak ezzel foglalkoztak elkezdettek aggódni hogy honnan lesz majd jövedelmük 
 Brassó volt az  aki 1600 körüli évek ideje alatt add hírt a a posztó kereskedelemről .
 Volt egy céhen kívül szervezet aki szét osztotta a mesterek között  posztót és az indigót , miután a mesterek el készültek a festetett mintás anyagokkal azokat összeszedték és ki vitték őket vársárokra 
 Az országnak azon északi részén ahol még nem voltak törökök 16-17 század környékén hirtelen fejlődés vette kezdetét 
 De akik menekültek a törökök elöl más falvakba és kezdtek menekülni és ezek a falvak már kicsit zsúfolttá váltak a termelés és maga a munka folyamat is több kezdet lenni .
 A kisvárosokban mint például Lőcse , Eperjes itt már a 16 században több mester  készített anyagok és fonal festést .
 AZ 1608-as évek idején sok vándorló volt és a magyarországi városokban más ott élő mesterekkel közösen hoztak létre festő céheket .
 Az elsö magyarországi feljegyzés kékfestésről 1783 ból van Körmöcbányáról 
 17 század közepe táján  meg is honosodott kékfestészet Magyarországon 
 18 században szinte minden városban volt olyan szak ember aki ezzel foglalkozott mert igen nagy kereslet vette a kezdettét nyomot mintás anyagok iránt sok mester próbált új technikákat létre hozni és elsajátítani amik már léteztek 
 Volt egy soproni család a 19 században  fontos szerpük volt mert ök voltak az elsö mesterek akik gyár jellegü kékfestö uzemet múgődtetek mint például a Goldberger műhely vagy volt mégegy ami szegeden volt  aki külföldi tapasztalatok alapján végzte ezt a mesterséget  aki a szegedi mühelyt vezet eaz Felmayer Antal volt aki 1826 ra gyárrá fejlesztette az egészet 
 Amikor megtörtént az ipari fejlödés egyszercsak maga a manufaktur és az ipar kezdet ugy mond a végéhez közeledni de próbáták még benne tartani az eröt .
A vidéki életformák változása után a városi dolgozoknak az öltözködésük is meg változott a régi hagyományos viseletek iényei is változni kezdtek magyarországon 
Az ipar legkiemelkedöbb mesetere egy székesfehérvári Felmayer István volt  kékfestö  aki egy kékfestö gyárat alapitott fiával együtt .
Aztán 1852 ben és 1861 ben  elkezte a bővitést , és a késöbbiekben külföldre exportálta a termékeket európa minen pontjára Indiába bérelte a földeket ahol saját indigot termesztett .
Az elsö és a másodig világháború után nyersanyag hiány lépet fel városiasodás csökentete a mühelyek számát 
De még egy két vidéki helyen ma is foglakoznak és viszik tovább ezt a mesterséget próbálják fent tartani ezt hogy ne veszen el a kékfestészetnek maga a varázsa .

 





\section{Technológiája}
Egy kékfestö műhely létrehozásához is jelentös tökebefektetésre volt szügség az iletönek  hiszen egy teljesen külöm épülett kellet hozzá 
Volt egy helység ami a szinezés és a kifözés helye volt ez volt az ún ezt nevezték még fekete konyhának is 
mintakészités is máshol végezték el az küpa szobának nevezték 
száritás az az udvaron folyt ha rosszab idö volt ezt akkkor bent kellet valahogy kivitelezni 
 ezzek a mühelyek átalábba a család telkére szokott kerülni igy nem véletelen a dolog hogy apárol fiura megy tovább ez a munka.
 A A nagyobb muhelyek rendelkeztek körübelül 4-5 katlanal .
 A vett vásznakad ki fözzték a a szenyezödések miatt , másfél óran keresztűl szódás vizben fözték .
 Mioözben ment a kifözés  keméynitös oldalát meg jelölték nehogy arra az oldara kezdjék el a mintázást mert sajnos atol hogy kifözték elöfordul hogy foltosan fogta be az anyagot 
 a fen maradt végeket feket konyhában volt körübelül egy literes fakád és abban és tiszat vizben ki öbliteték .
 Miután befejezödött a öblités jött a száritás amit az udvaron történt meg szárito az ugy nézet ki vagy az udvar közepén ált lábakon  vagy az épület oldalára volt erösitve . De mikor meg érkezett a fagyos idöszak aakor sajnos a kinti száritás nem volt jo anyagok számár mert a kemyényitö elkezdet elszínezödni az anyagon  leginkább ez az esett jobban a festet anyagokon látszott meg sajnos és azt már nem tudták felhasználni ez selejtek közé került , ebben az idöszakban szünetelteniuk kellet a munkát körübelül 6 hónap kimaradás volt a részükröl
Igy akkor kevesebb munkásra volt szügségük de addig az ugy mond pihenö idö alatt se tétlenkedtek hanem a minta ducokat tudták elökészitenni addig is .
 A századforduló idejére már több vidéki mühely is rendelkezet gözgéppel vagy kazánházzal ezzeknek a segitségével már betudtak szerelni télire száritokoat a műhelyen belüre is
 keményitésre egy gyenge búza , kukorica  vayy burgonyakemyényitésen ment keresztül a fekete konyha nagy kádjában.
 A megszáradt végeket elviték a Mángorlóba . ez az ugy nevezett mángorló egy erös gerendára szerelt asztalalpzatból ált rajat erös vastag keményfa görgökön mozgót mérete körübelül 4-6 méter hosszu lehete és körübelül két és félméter magas lehetet
 akkoriban ezzeket lovakal hajtoták .
 kapccsolodott hozzá egy láda a föllöt el haladt egy főtengely és arra került fel egy vastag lánc ami az asztalon lévö ládát mozgata .
 A belső kerekeknél két fötengely helyezkedet el váltokar segitségével lehetett irányitani ezzel azt tudták elérni hogy jobbra és balra is tudták forgatni a szerkezetett
 és plusz az volt hogy kis helyen is elvéghetö legyen ez a folyamat a munka során 
 elöfordult olyan is amikor visza esett egy kicsit a az üzlett forgalam es akkor sajnos nem volt pénz lóra aki hajtsa a kereket akkor maga a mester áltbe és tekerte a szerkezetett
 A mángorló több változta is létezik az általános méret 30x30  ezzeket a szerkezetek általában vastag gerendákból épitetteék fel 15-20mm vaslemezeket szereltek fel rá .
 Azokba a mühelyekben ahol adott volt a gépesités ott gőzgép vagy transzmissziós tengelyekkel vitte az eröt a mángorlóba ezt a technólógiát helyileg ,Pápán ,Csornán Békéscsabán használták 
 más varosokban ( Bátaszék) elöfordult  hogy itt benzimotort alkalmaztak Csornán villanymeghajtásura épitették át a mángorlót 
 Fontos alkatrésze a mángorlásnak a görgö, és a áruhenger .
 Az únt felhelyzték a felsodrószékre  és a vászon végét beburkolták .
 Miközben mozgásban volt nagy súly volt rajta ide oda mozgott és eközzben ki simitotta az anyagott 
 Mielött elkezdödött a mintázás kétszer háromoszor végezték el ezt a folyamatott festés után a kemyényités körübelül hat - nyolc-tiz alkalomal végezték el 
 Maga a mángórlo téglával lerakott helység volt itt készitették el a vegyszereket is .
 Amit a mintakészitésnél hassznált fö fedő anyag az a pap volt áltlában más - más öszetételben készült ez anyag nagy titkos receptek alapján készitették el 
 és hogy több hónapig elegendö legyen igy egyszere egy nagy menyiségek készitettek el 
 Ahhoz hogy ne mennyen tönkre ez az anyag nyirkos hüvös helyen tartották 
 ezt az anyagot külümbözö pácokhoz és fedőanyagokhoz is hozzá keverték mikor megkezdödött akék festés akkor mészlúgós öblitésel sárga zöld , narncs szinű mintákat is tudtak elöállitani
 A mintázáshoz is volt egy külön helység amit használtak és ott abban a helységben volt a tarkázóasaztal e mellet ehlyezkedet el a a sasi 
 A Sasi egy olyan ládad volt ammi körübelül 6-8 cm mélységü volt és benne  egy sürü keményitó volt
 a ládában volt egy keret aminek ez egyik oldala viaszosvászonal a mésik rész epedig monilóval van be vonva ebben kenik el a fedömasszát a papot  és kürübelül ugy müködik mint a pecsét párna .
 A minta készités nyomoducokal folyt volt méret külömségek a ducok közzöt de ahhoz hogy hatékonyabb legyen  mintás anyagok készitése nagyom ducokre kellet válltani de azzoknak az elkészitési ideje sok idött vett  igénybe igy átt váltottak egy  ugy nevezett perrotingépre ami felgyorsitotta a gyártást .
 1860 körül járhatunk amikor  ugyan is bekövetkezett a századfordulo és a gépiesités igy egyre több mühely átt álltak a gépi nyomásra .
 










 






\subsection{Alapanyaga}
Az elsö alapanyag amit felhaszna kaz maga a festö anyagok 



\subsection{Folyamata}
kékfestészet  munkafolyamata ugy zajlott le egy műhelyben . hogy a nyersanyagot magánszemélytöl vásárolták vagy maga rendelö vitte magával az anyagot és még a mintát is ki tudta választanoia  a mintakonyvböl
Budapesten voltak olyan emberek hogy csak azzal foglalkoztak a kékfestésel foglakozo mesterekt látták el anyagókal
Az első  folyyamt amit kezdenek az a anyagokal .hogy ki mossák öket szódás vízben ez körübelül két órát vesz igénybe .
 A szódás kifözést azért végzik el hogy az anyagböl ki oldoldjon egy gyantás anyag és ez a folyamat után  az anyag kifehéredik
 Miután ez a folyamt befejezödött akkor következik a hideg vizes öblités majd léces száritora helyezik majd néhány óra alatt meg szárad ez a folyamt nyáron gyorsanbb télen egy kicsit lassab folyamat 
Mikor már megszáradt az anyag egy kap egy vékony keményitést ami elkökésziti a mintázáshoz 
Mintázás régies nevén Tarkázás .
Tarkázásnak emlitett kézi minta készités  párnázott asztalon készül.
Amire több rétegben pokrocokat helyeztek majd a végén molinoval befedték igy az asztal felülete 
Igy az asztal felülete puha lett .
Az asztal mellet helyezték el Únt és az ún-ba mártoták bele a minta ducokat.
A viaszosvászon belsö felületére kent fel a mintaázó pépet az Ún 
ládát feltőltőték sürü keményitövel ami a sasit puhán tartotta .
Amit általábban használunk hozzá  fehér nyomo pépet annak alkotorészei ólómnitrát, ólomacetát . gumiarábikum . kékő.timsó, rézgálic .
Ezzeket össze fözzik és miután kihült egy hétre rá használják csak az anyagot 
A festék pépet  maga a mester szokta elkészíteni általában ez a festék pép akkár ávekig is elég szokot lenni a mester az gondolják hogy egy pép minél régebbi egy nyomó pép annál fehérebbé teszi a mintát
A kékfestő a minta  dúcot a nyomópépbe nyomja majd és a dúcot vászonra illeszti pontosan nehogy elcsúszon és majd erősen lenyomja hogy minden pontra kerüljön a pépböl és azt a részt nem fogja be az indigófesték .
Mintázás nagy figyelmet és türelmet igényelt 
Általában a kékfestők  ,120-130 méter anyagot meg mintáztak de az napi 13-14 órás munka folyamat volt -
Miután a minta nyomás befejeződött  párnap volt még megszáradt a nyomatés ezek után következet a festés .
Az anyagot egy ráfra akasztják   amÍ  igazából egy vaskerék érre akasztják fel az anyagot és és úgy nézi ez az anyag a ráfon és olyan az esése az anyagnak ezen a vaskereken mintha egy rakott szoknya lenne .
következő teendő pedig nem  más  lengedik  az indigó festékkel teli kád szerűségbe és  félóránkként fel engedik és körülbelül - percet hagyják a levegőn hogy az indigó oxidálódjon .
Amikor elsőnek felhúzzák az anyagot a festékből akkor ilyen sárgás zöld színe van aztán megint eltelt egy kis idő akkor ilyen sötétzöld színe lesz az útólö előtti merítés után világos kék színe lesz 
miután egyre többet éri a levegő  megkapja azt a szép sötétkék színt 
Miután az anyag elérte a kívánt színt akkor következik a száritás.
 Ezután miután meg száradt akkor kap egy kénsavas fürdőt ennél a folyamatnál az történik hogy kioldódióik a mintázásnál beleragadt nyomópép és ezek után nagyon szépen elő jön a minta és utána egy sima vízben is ki öblítik  és hagyják meg száradni , ki keményítik az anyagot .
 Ez volt a teljes munkafolyamat .
 

















\section{A természet mint inspiráció}


\section{Minta dúcok}
Maga a minta és a forma készítés rajztudást igényelt és a textilnyomáshoz való ismereteket igényelt 
mintáknak mindig pontosan kellet egymáshoz illeszkedniük  ezt a készítőnek mindig tudnia kellet .
A nyomódúcok felületei általában mindig kiemelkedtek és negatívba kellet rajtuk a mintákat elkészíteni 
Két fajta minta dúcot készítünk fából kifaragott vagy réz lemez formájú mintát 
első az hogy  fából kifaragott mintákat ki rajzolják a mintafára utána azt elkezdik kifaragni 
a következő fajta úgy készült hogy a fatáblát kivésték  rézlemezek helyét fémlemezeket a helyükre tették és utána vízbe áztatták 
A dúcok hátlapjára egy erős fa lemezt erősítettek és ez a fa lemez segítségével könnybe volt a nyomdázás
A kedvenc és bevált mintákat idomait mintázó gép nyomó fáira is átvitték .
Igy ma készített minták ornamentikája őrzik a több évtizedes mintákat és ízlés hagyományokat.
A kezdetekor úgy nevezték a kézi minta rajzokat mint például  ágas, ágasindás , ágasvirágos, csokros, cakkos ,folyondár , margarétás , négyzetpontos ,
akkoriban nagyon egyszerű neveket adtak a kész mintáknak .
A 17 század és a 18 századi minta dúcokon pozitív  nyomást alkalmaztak amikor növényi vagy figurális mintákat készítettek ezeket a fa metszeték  dúcokra készítették.
Ugyan ebben az időszakban már szinte minden városban volt egy minta készítő mester .
Sasvári a cseklésziek  a 18 század közepe táján ők több minta készítő mestert alkalmaztak azért volt annyi emberük hogy minél gyorsabban el tudják készíteni a mintáikat 
még az itteni munka mellet tudtak foglalkozni vidéki műhelyek mintáival de miután végeztek mentek is tovább .
Miután elkezdtek terjedni nyomot kelmék igy vidéki műhelyek mintakincsei is kezdtek felkapotabbá  válni .
Ez idő alatt külföldről be érkező emberek letelepedtek egy adott időre az adott mestereknél akinek elkészítette a mintáját és addig kapott ellátást neki csak a munkával kellet foglalkozni miután végzett ment is tovább 
Ez idő alatt Magyarországon csak ez a kettő minta készítő műhely volt már korábban említett Sasvári és Cseklészi műhelyek vagy úgy nevezett manufaktúrák voltak .
Későbbiek ben jöttek létre nagyobb gyárak mint például Goldbergerés Spitzer Gerson és aSzegedi Felmayer ők alkalmaztak mintafaragó és készítő szakembereket 
legelső minta fák teljes mértékben fából készültek kisebb pontok, ágak , hasonló képen mint a folt felületek de ezeknek a készítésére legalkalmasabbnak a dió fát tartották vagy körte és gesztenye fa mert ezek a fajták nem szálkásodtak  egyszerű volt ki faragani belőlük a mintákat tartósak voltak és  nem repedeztek ki .
A 16  század idején néhány nyomó dúc kőzött elfordult tölgy és  bükk is ezeknek használatával sok változás jött elő
ugyan is ezeknél kezdték el használni a réz drótot de későbbiekben már teljes mértékben vörös réz drótot használtak
azért volt jobb csak a vörös réz drótot használni mert az előállítása olcsóbb volt és egyre több mintát tudták elkészíteni 
A fémből készített mintáknak a legfontosabb tartozéka az ún volt .
A fémből készült minták nagyon időigényesek voltak 
A mintákat úgy kezdik el megtervezni hogy elszőr papír terveket készítenek miután a elfogadták a mintát akkor elkezdi felrajzolni a minta fára és utána be lakozza hogy miközben készít ne mosódjon el rajt a minta .
ritkább vagy bonyolultabb mintáknál a be karcolják a a mintafát hogy pontosan helyezkedjenek el az elemek .
A minta dúcokra készített mintákhoz lapos vésőt ütögetve haladnak  és közben beverik a a lemezeket .Ha több szint szeretek volna alkalmazni akkor egy kiegészítő dúcot alkalmazták 
A békéscsabai festő műhely tulajdonosa Sztaricskay Pál  régi kedven kézi mintáit elkészítette a gépi nyomáshoz is .
Még aki gépi mintákat készített az Gál Gyula volt és ö még adott is el belőlük .







\subsection{Alap minták}
AZ általános mintáknak a virágosokat lehetne mondani 
\subsection{Minta fejlődés}
\subsection{Egyébb fő minták motívumok}
\section{Hagyományos kékfestő műhelyek}
\section{Egyéb műhelyek}
\subsection{Kortás megoldások}
\subsection{Adaptációk}

\chapter{Összefoglalás} 


\printbibliography
\renewcommand{\appendixname}{Függelék}  % különben "Appendix"-et ír
\appendix
\chapter{Mellékletek listája}

\end{document}